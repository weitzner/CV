\documentclass[12pt]{article}

%\usepackage[T1]{fontenc}
%\usepackage{mathpazo}
\usepackage{fullpage}
\usepackage{amsmath,amsfonts}
\usepackage{textgreek}

\usepackage{fontspec,xltxtra,xunicode}
\defaultfontfeatures{Mapping=tex-text}
%\setromanfont[Mapping=tex-text]{Hoefler Text}
%\setromanfont[Mapping=tex-text]{Bembo Book MT Pro}
\setromanfont[Mapping=tex-text]{Minion Pro}
\setsansfont[Scale=MatchLowercase,Mapping=tex-text]{Gill Sans}
\setmonofont[Scale=MatchLowercase]{Andale Mono}

%% Fixes for letterspacing that work with XeLaTeX
\newcommand{\allcapsspacing}[1]{{\addfontfeature{LetterSpace=7.5}#1}}
\newcommand{\smallcapsspacing}[1]{{\addfontfeature{LetterSpace=5.0}#1}}
\renewcommand{\textsc}[1]{\smallcapsspacing{\scshape{#1}}}

\usepackage{CV}
%\usepackage[letterspace=100]{microtype}

\newcommand{\tworowstwocolsitem}[4]{
\noindent
\begin{tabular*}{\textwidth}{@{\extracolsep{\fill}}lr}
		#1 & #2 \\
		#3 & #4 \\
\end{tabular*}\vspace{0.75\baselineskip}}

\newcommand{\educationitem}[4]{
\tworowstwocolsitem{#1}{#2}{\textit{#3}}{\textit{#4}}}

\newcommand{\teachingitem}[4]{
\tworowstwocolsitem{#1}{#2}{\textit{#3}}{#4}}

\newcommand{\leadershipitem}[3]{
\noindent
\begin{tabular*}{\textwidth}{@{\extracolsep{\fill}}lr}
		#1 & #2 \\
\end{tabular*}\newline#3\vspace{0.75\baselineskip}\par}

\newcommand{\outreachitem}[4]{
\noindent
\begin{tabular*}{\textwidth}{@{\extracolsep{\fill}}lr}
		#1 & #2 \\
		#3 & \\
\end{tabular*}\newline#4\vspace{0.75\baselineskip}\par}

\newcommand{\researchitem}[4]{
\outreachitem{#1}{#2}{\textit{#3}}{#4}}

\newcommand{\yearitem}[2]{
\noindent
\begin{tabular*}{\textwidth}{@{\extracolsep{\fill}}lr}
		#1 & #2 \\
\end{tabular*}\vspace{0.75\baselineskip}}

\usepackage{titlesec}
\titleformat{\section}{}{}{}{\hrule\vspace{0.5\baselineskip}\sffamily\Large\uppercase}{}{}
\titlespacing{\section}{0pt}{-15pt}{*1}

\titleformat{\subsection}{}{}{}{\sffamily\large\uppercase}{}{}

% enables reverse numbering
\usepackage[leftmargin=12pt, itemsep=0.5\baselineskip]{etaremune}

\begin{document}

\begin{center}
\Large{\sffamily\allcapsspacing{\uppercase{Brian D. Weitzner}, Ph.D.}}

\normalsize{\textsc{Department of Chemical \& Biomolecular Engineering \\
Johns Hopkins University \\
3400 N. Charles Street, Baltimore, Maryland 21218 \\
\vspace{0.25\baselineskip}
brian.weitzner@jhu.edu}}
\end{center}

\section*{\allcapsspacing{Education}}
\educationitem{The Johns Hopkins University}{Baltimore, MD}{Ph.D. Chemical \& Biomolecular Engineering}{March 2015}
\educationitem{Cornell University}{Ithaca, NY}{B.S. Chemical \& Biomolecular Engineering, Minor Biomedical Engineering}{2009}

\section*{\allcapsspacing{Research Experience}}
\researchitem{\textbf{Postdoctoral Fellow}, The Johns Hopkins University}{March 2015--Present}{Advisor: Dr. Jeffrey J. Gray}{\par\noindent
Developed a new method for \textit{de novo} CDR H3 loop structure prediction using structure-based constraints.
% Redesigned and implemented an all-new codebase for RosettaAntibody.
}
\researchitem{\textbf{Graduate Research Assistant}, The Johns Hopkins University}{August 2009--March 2015}{Advisor: Dr. Jeffrey J. Gray}{Dissertation title: Next-generation antibody modeling.
\vspace{0.75\baselineskip}\par\noindent
My studies focused on antibody structure prediction, namely accurate \textit{de novo} modeling of the CDR H3 loop.
I determined the weaknesses in existing structure prediction methods during a blind structure prediction challenge.
To rectify these weaknesses, I performed statistical analyses of known structures of antibodies and non-antibody proteins to determine what governs the conformation of CDR H3 loops.
Using the results from these analyses, I formulated a structure-based constraint that can be applied during \textit{de novo} modeling to produce accurate models of CDR H3 loops.
During this work I developed a new hypothesis regarding the functional utility of the C-terminal kink present in most antibody CDR H3 loops.}
\researchitem{\textbf{Undergraduate Research Assistant}, Cornell University}{November 2006--May 2009}{Advisors: Dr. Matthew P. DeLisa \& Dr. Jeffrey D. Varner}{Computational retargeting of an E3 ubiquitin ligase}
\researchitem{\textbf{Undergraduate Research Assistant}, Fox Chase Cancer Center}{June 2005--August 2009}{Advisor: Dr. Roland L. Dunbrack, Jr.}{Dimerization motifs of cytosolic sulfotransferases}
\researchitem{\textbf{Howard Hughes Student Scientist}, Fox Chase Cancer Center}{September 2004--June 2005}{Advisor: Dr. Roland L. Dunbrack, Jr.}{Agreement among automated quaternary structure prediction methods}

\section*{\allcapsspacing{Publications}}
\begin{etaremune}
\item Porter JR, \textbf{Weitzner BD}, Lange OF (2015) ``A framework simplifying combined sampling modes in Rosetta,'' \textit{under review}
\item Alford RF*, Koehler Leman J*, \textbf{Weitzner BD}, Duran AM, Tilley DC, Elazar A, Gray JJ (2015) ``An integrated framework advancing membrane protein modeling and design,'' \textit{PLoS Comput. Biol.} In press. (* equal contribution authors)
\item \textbf{Weitzner BD}, Dunbrack RL, Jr, Gray JJ (2015) ``The origin of CDR H3 structural diversity,'' \textit{Structure}, 23(2), 302--11.
\item \textbf{Weitzner BD*}, Kuroda D*, Marze N, Xu J, Gray JJ (2014) ``Blind prediction performance of RosettaAntibody 3.0: Grafting, relaxation, kinematic loop modeling, and full CDR optimization,'' \textit{Proteins} 82(8), 1611--23. (* equal contribution authors)
\item Lyskov S, Chou F-C, Conch{\'u}ir S{\'O}, Der BS, Drew K, Kuroda D, Xu J, \textbf{Weitzner BD}, Renfrew PD, Sripakdeevong P, Borgo B, Havranek JJ, Kuhlman B, Kortemme T, Bonneau R, Gray JJ, Das R (2013) ``Serverification of Molecular Modeling Applications: The Rosetta Online Server That Includes Everyone (ROSIE),'' \textit{PLOS ONE} 8(5): e63906.
\item Baugh EH, Lyskov S, \textbf{Weitzner BD}, Gray JJ (2011) ``Real-time PyMOL visualization for Rosetta and PyRosetta,'' \textit{PLOS ONE} 6(8): e21931.
\item Chaudhury S, Berrondo M, \textbf{Weitzner BD}, Muthu P, Bergman H, Gray JJ (2011) ``Benchmarking and analysis of protein docking performance in Rosetta v3.2,'' \textit{PLOS ONE} 6(8): e22477.
\item Bourne PE, Beran B, Bi C, Bluhm W, Dunbrack R, Prlic A, Quinn G, Rose P, Shah R, Tao W, \textbf{Weitzner B}, Yukich, B (2010) ``Will Widgets and Semantic Tagging Change Computational Biology?'' \textit{PLoS Comput. Biol.} 6(2): e1000673.
\item \textbf{Weitzner B}, Meehan T, Xu Q, Dunbrack R (2009) ``An unusually small dimer interface is observed in all available crystal structures of cytosolic sulfotransferases,'' \textit{Proteins.} 75(2), 1097--134.
\end{etaremune}

\section*{\allcapsspacing{Funding}}
\noindent Contributed to and secured \textbf{NIH 5R01-GM078221}, ``Prediction of the structure of therapeutic antibodies with their antigens,'' 9/1/2012--8/31/2016, \$1,241,054 (\$821,600 direct).
Specifically responsible for the development of Aim 1: Improve prediction of long, hypervariable CDR H3 loops.
\vspace{0.75\baselineskip}

\noindent Work toward this aim led to two first-author publications: (1) ``The origin of CDR H3 structural diversity'' with RLD, and JJG; and (2) ``Blind prediction performance of RosettaAntibody 3.0: Grafting, relaxation, kinematic loop modeling, and full CDR optimization'' with DK, NM, JX, and JJG.
\vspace{0.75\baselineskip}

% some items use leadership item instead of year item to make the year appear in line with the top line for multiline entries
\section*{\allcapsspacing{Selected Honors and Awards}}
\yearitem{RosettaCon XIII Best Poster Award}{2015}
\yearitem{JHU ChemBE Department Graduate Student Award}{2015}
\yearitem{Rosetta Service Award: Instructor at inaugural Rosetta Boot Camp}{2013}
\leadershipitem{Rosetta Service Award: Leader of transition of Rosetta source}{2013}{code to a new version control system}
% \yearitem{Rosetta Service Award: Leader of transition of Rosetta source\\code to a new version control system}{2013}
\yearitem{American Institute of Chemists Student Award}{2009}
\leadershipitem{1$^\text{st}$ place in national AIChE Car Competition;}{2008}{first team to ever perform perfectly}
%\yearitem{1$^\text{st}$ place in national AIChE Car Competition;\\first team to ever perform perfectly}{2008}
\leadershipitem{Howard Hughes Medical Institute Student Scientist Program;}{2004--2005}{Fox Chase Cancer Center}
%\yearitem{Howard Hughes Medical Institute Student Scientist Program;\\Fox Chase Cancer Center}{2004--2005}
\yearitem{Eagle Scout}{2003}

\section*{\allcapsspacing{Invited Seminars and Talks}}
\begin{etaremune}
\item \textbf{Weitzner BD}, Gray JJ (2015) ``Next-generation Antibody Modeling'' \textit{Seminar, Institute for Cellular and Molecular Biology, University of Texas at Austin}, Austin, TX.
\item \textbf{Weitzner BD}, Gray JJ (2014) ``Next-generation Antibody Modeling'' \textit{Seminar, Center for Biomolecular Structure and Dynamics, University of Montana}, Missoula, MT.
\item \textbf{Weitzner BD}, Dunbrack RL, Gray JJ (2014) ``The origin of CDR H3 Structural Diversity'' \textit{Vortrag, Fakult\"{a}t f\"{u}r Chemie, Technische Universit\"{a}t M\"{u}nchen}, Munich, Germany.
\item \textbf{Weitzner BD}, Gray JJ (2013) ``Computational Structure Prediction, Docking and Design of Antibodies'' \textit{IBC Antibody Engineering and Therapeutics}, Huntington Beach, CA. (delivered on behalf of JJG during his paternity leave)
\end{etaremune}

\section*{\allcapsspacing{Conference Contributions}}
\subsection*{\allcapsspacing{Talks}}
\begin{etaremune}
\item \textbf{Weitzner BD}, Kuroda D, Marze N, Xu J, Gray JJ (2013) ``Benchmarking RosettaAntibody: Antibody Modeling Assessment II'' \textit{Antibody Engineering and Therapeutics Conference}, Huntington Beach, CA.
\item \textbf{Weitzner BD}, Roland RL, Gray JJ (2013) ``Kinked CDR H3-like loops are common'' \textit{AIChE Annual Conference}, San Francisco, CA
\item \textbf{Weitzner BD}, Dunbrack RL, Gray JJ (2013) ``Antibodies are proteins too!'' \textit{Rosetta Conference}, Leavenworth, WA.
\item \textbf{Lyskov S}, \textbf{Weitzner BD}, Gray JJ (2011) ``PyRosetta 2.0: I can make a new score term in 6 lines!'' \textit{Rosetta Conference}, Leavenworth, WA.
\item \textbf{Weitzner BD}, Leaver-Fay A, Kulp D,  Lyskov S (2010) ``Using PyRosetta for research'' \textit{Rosetta Conference}, Leavenworth, WA. [workshop]
\item \textbf{Weitzner BD}, Baugh EH, Gray JJ (2010) ``PyMOL--PyRosetta Integration'' \textit{Rosetta Conference}, Leavenworth, WA.
\end{etaremune}

\subsection*{\allcapsspacing{Posters}}
\begin{etaremune}
\item Weitzner, BD, Dunbrack RL, Gray JJ (2015) ``\textit{De novo} CDR-H3 loop structure prediction using a structurally derived kink constraint'' \textit{Rosetta Conference}, Leavenworth, WA.
\item Weitzner, BD, Dunbrack RL, Gray JJ (2015) ``The origin of CDR H3 Structural Diversity'' \textit{Biophysical Society Meeting}, Baltimore, MD.
\item Weitzner, BD, Dunbrack RL, Gray JJ (2014) ``CDR H3 loop prediction'' \textit{Rosetta Conference}, Leavenworth, WA.
\item Weitzner, BD, Dunbrack RL, Gray JJ (2012) ``Are CDR H3 loops special?'' \textit{Rosetta Conference}, Leavenworth, WA.
\item Weitzner, BD, Dunbrack RL, Gray JJ (2011) ``Accessing the conformation space of long CDR H3 loops through \textbeta-turn detection'' \textit{Rosetta Conference}, Leavenworth, WA.
\end{etaremune}

\section*{\allcapsspacing{Teaching Experience}}
\teachingitem{Guest Lecturer, ChemBE 414/614}{Fall 2014}{Computational Protein Structure Prediction and Design}{Johns Hopkins University}
\teachingitem{Co-Instructor, Rosetta Boot Camp}{Spring 2013}{An intense week-long crash course to developing in Rosetta}{Chapel Hill, NC}
\teachingitem{Co-Instructor, ChemBE 418}{Fall 2011}{Projects in the Design of a Chemical Car}{Johns Hopkins University}
\teachingitem{Teaching Assistant, ChemBE 409}{Fall 2010}{Modeling, Dynamics \& Control of Chemical \& Biological Systems}{Johns Hopkins University}
\teachingitem{Teaching Assistant, ChemBE 414/614}{Spring 2010}{Computational Protein Structure Prediction and Design}{Johns Hopkins University}
\teachingitem{Teaching Assistant, ChemE 3900}{Spring 2009}{Chemical Kinetics and Reactor Design}{Cornell University}
\teachingitem{Teaching Assistant, ChemE 1120}{Fall 2008}{Introduction to Chemical Engineering}{Cornell University}

\section*{\allcapsspacing{Scientific Leadership}}
\leadershipitem{Developed RosettaCon Code of Conduct}{2014}{Led the development and implementation of Code of Conduct for RosettaCon to promote diversity. Served on the incident reporting panel 2014--2015.}
\leadershipitem{Rosetta Boot Camp}{2013}{Identified need for training graduate students and postdocs. Helped developed curriculum to introduce students to Rosetta.}
\leadershipitem{Undergraduate research mentor}{2012--2014}{Mentor to four undergraduate researchers in the Gray Lab}
\leadershipitem{Organizer, Rosetta Developer Meeting}{2012}{The annual Rosetta Developer Meeting was held in Baltimore. Arranged the program, travel, lodging and meals}

\section*{\allcapsspacing{Activities and Outreach}}
\leadershipitem{Student volunteer, STEM Achievement in Baltimore Elementary Schools}{2013--2015}{Monthly visits to a Baltimore City elementary school to facilitate learning STEM skills}
\leadershipitem{Runner, Baltimore Marathon}{2012}{\vspace{-\baselineskip}}
\leadershipitem{Member of the Extreme Rosetta Workshop (XRW) Team}{2010--2011}{A small team of developers gathered to overhaul the structure of the Rosetta source code}
\leadershipitem{Volunteer at the Ricky Myers Day of Service}{Fall 2010}{A city-wide day of service to clean parks, plant gardens, repair homes and more}
\leadershipitem{Member of the JHU ChemBE Department STEM outreach group}{2009--2013}{Visits to a Baltimore Recreation Center to teach children about STEM through demonstrations and activities}
\leadershipitem{Captain of the AIChE Car Team, Cornell University}{2008--2009}{Leader of the team, organized various sub-groups and kept the project on schedule}
\leadershipitem{Member of the AIChE Car Team, Cornell University}{2006--2008}{A project team that builds a shoe box-sized car that is powered and stopped by chemical reactions}

\end{document}
